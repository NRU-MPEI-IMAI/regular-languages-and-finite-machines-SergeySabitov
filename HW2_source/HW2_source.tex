\documentclass[a4paper,12pt]{article}
 
\usepackage{cmap}		
\usepackage[utf8]{inputenc}			
\usepackage[english,russian]{babel}
\usepackage{framed}
\usepackage{hyperref}
\usepackage{amsmath}
\usepackage{graphicx}
\usepackage[colorinlistoftodos]{todonotes}
\usepackage{wrapfig}
\usepackage{lipsum}
\usepackage{color}
\usepackage{indentfirst}
\usepackage{times}
\usepackage{textcomp}
\usepackage{float}
\usepackage{listings}
\usepackage{xcolor}
\usepackage[T1]{fontenc}
 
\usepackage{morewrites}
 
\usepackage[pdf]{graphviz}
\usepackage{xpatch}
\makeatletter
\newcommand*{\addFileDependency}[1]{% argument=file name and extension
  \typeout{(#1)}
  \@addtofilelist{#1}
  \IfFileExists{#1}{}{\typeout{No file #1.}}
}
\makeatother
\xpretocmd{\digraph}{\addFileDependency{#2.dot}}{}{}
 
\usepackage[autosize]{dot2texi}
\usepackage{tikz}
\usetikzlibrary{shapes,arrows}

\title{Домашнее задание №2}
\author{Сабитов Сергей А-13а-19}

\begin{document}
\maketitle


\section{Задание №1}
\subsection{$L = \{\omega \in \Sigma* |$  w содержит подстроку $aa\}$}

$$ S \rightarrow aS \:|\: bS \:|\: cS \:|\: aaT $$
$$ T \rightarrow aT \:|\: bT \:|\: cT \:|\: \lambda$$

\subsection{$L = \{\omega \in \Sigma* |$  w не палиндром\}}
$$S \rightarrow aS(a|b|c)\; | \; bS(a|b|c)\; | \;cS(a|b|c) \; | \; aTb \; | \; aTc\; | \; bTa\; | \;bTc \; | \;cTa \; | \;cTb \; $$
$$ T \rightarrow aT \; | \; bT \; | \; cT \; | \; a \; | \; b \; | \; c \; |\; \lambda$$

\subsection{$\Sigma = \{ \emptyset, N, `\{`,`\}`, \cup \} $ }
Постройте грамматику для языка $L = \{\omega \in  \sum^*|\omega $ - синтактически корректная строка, обозначающая множество $\}$ 


$$S \rightarrow Z\; | \;Z\: u\: S \;|\; \{Z\: u\: S\}$$
$$Z \rightarrow  R\; |\; \{\:\} \;|\; \{\:Z\:\}\; |\; \{\:Z,\:Z\:\}$$ 
$$R \rightarrow  T\:,\:R \;|\;T$$
$$T \rightarrow  N\:,\:O\;|\;O\:,\:N\;|\;N\;|\;O$$

\section{Упражнение №2}
 
    $A = \{1^m + 1^n = 1^{m+n} :\ | :\ m, n \in \mathbb{N}\}$
 
    \subsection{Докажите, что язык $A$ регулярный (построением) или нерегулярный (через лемму о накачке)}
 
        Будем доказывать, что язык нерегулярный:
 
        \begin{enumerate}
            \item Фиксируем $n = m + l + 2$
            \item Берем $w = 1^m + 1^{l+1} = 1^{m+l+1}$
            \item $|w| = 2(m + l) + 2 \geq n$
            \item Рассмотрим разбиение:
 
            $x = \{1^m+\}$
 
            $y = \{1^{l+1}\}$
 
            $|xy| = m + l + 1 \leq n; \: |y| = l + 1 \geq 1$
 
             $z = \{=1^{m+l+1}\}$
 
            \item $\forall k \geq 0: xy^kz \in L$ - не выполняется, так как при $k = 0 \; k \geq 2 \Rightarrow 1^m + 1^{k(l+1)} = 1^{m+l} \Rightarrow m + kl + k \neq m + l + 1$.  Следовательно, язык нерегулярный.
        \end{enumerate}
 
    \subsection{Постройте КС-грамматику для языка $A$, показывающую, что $A$ - контекстно-свободный}
 
        \begin{itemize}
            \item $S \to += \: | \: 1+=1 \: | \: +1=1 \: | \: 1S1 \: | \: 1+1T11$
 
            \item $T \to 1T1 \: | \: =$
        \end{itemize}


\begin{center}
    $L_11 = \{\Sigma = \{a, b\}, Q_1 = \{1, 2, 3\}, 1, T_1 = \{3\}, \delta_1 \}$
 
    \digraph{0211}{
        size="6,6";
        rankdir="LR";
        node [shape=point]; 0;
        node [shape=circle]; 1 2;
        node [shape=doublecircle]; 3;
        0 -> 1;
        1 -> 1 [label="b"];
        1 -> 2 [label="a"];
        2 -> 2 [label="b"];
        2 -> 3 [label="a"];
        3 -> 3 [label="a, b"];
    }
 
    $L_12 = \{\Sigma = \{a, b\}, Q_2 = \{1, 2, 3\}, 1, T_2 = \{3\}, \delta_2 \}$
 
    \digraph{0212}{
        size="6,6";
        rankdir="LR";
        node [shape=point]; 0;
        node [shape=circle]; 1 2;
        node [shape=doublecircle]; 3;
        0 -> 1;
        1 -> 1 [label="a"];
        1 -> 2 [label="b"];
        2 -> 2 [label="a"];
        2 -> 3 [label="b"];
        3 -> 3 [label="a, b"];
    }
\end{center}
\section{Упражнение 3}
 
    \subsection{Прогулка с поводком}
 
        Пусть $D_1 = \{\omega \in \Omega^* \: | \: \omega$ описывает последовательность ваших шагов и шагов вашей собаки на прогулке с поводком $\}$.
 
        \begin{enumerate}
            \item Докажите, что язык $D_1$ регулярный (построением) или нерегулярный (через лемму о накачке)
 
            Построим ДКА, тем самым, покажем, что язык регулярный:
            \begin{center}
                \digraph{g}{
                    size="6,6";
                    rankdir="LR";
                    node [shape=point]; 0;
                    node [shape=circle]; 2 3 4 5;
                    node [shape=doublecircle]; 1;
                    0 -> 1;
                    1 -> 2 [label="h"];
                    2 -> 1 [label="d"];
                    2 -> 3 [label="h"];
                    3 -> 2 [label="d"];
                    1 -> 4 [label="d"];
                    4 -> 1 [label="h"];
                    4 -> 5 [label="d"];
                    5 -> 4 [label="h"];
                }
            \end{center}  
 
            \item Постройте КС-грамматику для языка $D_1$, показывающую, что $D_1$ - контекстно-свободный
 
                \begin{itemize}
                    \item $S \to hT \: | \: dR \: | \: \lambda$
 
                    \item $T \to hdT \: | \: dS$
 
                    \item $R \to dhR \: | \: hS$
                \end{itemize}
 
 
        \end{enumerate}
 
    \subsection{Прогулка без поводка}
 
        Пусть $D_2 = \{\omega \in \Omega^* \: | \: \omega$ описывает последовательность ваших шагов и шагов собаки на прогулке без поводка $\}$.
 
        \begin{enumerate}
            \item Докажите, что язык $D_2$ регулярный (построением) или нерегулярный (через лемму о накачке)
 
            С помощью леммы о накачке покажем, что язык нерегулярный:
            \begin{enumerate}
                \item Фиксируем $n$
                \item Берем $w = h^n d^n$
                \item $|w| = 2n \geq n$
                \item Рассмотрим разбиение:
 
                $x = h^i$
 
                $y = h^j$
 
                $|xy| = i + j = n; \: j > 0$
 
                $z = h^{n-i-j}d^n$
 
                \item $\forall k \geq 0: xy^kz \in L$ - не выполняется, так как при $k \geq 2 \Rightarrow h^{kj+n-j}d^n \rightarrow h^{n+j(k-1)}d^n$ следовательно, человек и собака не будут в одной точке. Значит, язык нерегулярный.
            \end{enumerate}
 
            \item Постройте КС-грамматику для языка $D_2$, показывающую, что $D_2$ - контекстно-свободный
 
                \begin{itemize}
                    \item $S \to dShS \; | \; hSdS \; | \; \lambda$
                \end{itemize}
 
 
        \end{enumerate}
  
\section{Задание №5.}
\subsection{ Привести алгоритм построения НКА по праволинейной грамматике. Доказать, что с помощью алгоритма мы можем получить только слова из языка грамматики. Проиллюстрировать алгоритм на грамматике:}
\begin{center}
    $A \rightarrow aB|bC$
    
    $B \rightarrow aB|\lambda$

    $C \rightarrow aD|A|bC$

    $D \rightarrow aD|bD|\lambda$
\end{center}

{Алгоритм: }
\begin{itemize}
 \item {
    Множество   вершин   НКА   состоит   из   нетерминалов грамматики  и,  возможно, еще  одной  новой  вершины F, которая объявляется заключительной.
 }
 \item {
    Каждому  правилу  вида  A$\rightarrow$aB  в  автомате  соответствует дуга  из  вершины А в  вершину В,  помеченная  символом а.  Каждому  правилу  вида  A$\rightarrow$a  соответствует  дуга из  вершины А в  вершину F,  помеченная  символом а. Других дуг нет.
 }
 
 \item {
 Начальной вершиной   автомата   является   вершина, соответствующая    начальному    символу    грамматики. Заключительными    являются новая вершина F, если  она использовалась на шаге 2, и каждая вершина A, такая что для нетерминала A в грамматике есть правило  $A \rightarrow \lambda$
 }
 
\end{itemize}

{Допустим, что наш алгоритм строит автомат, который допускает слова, которых нет в языке. Тогда существует переход от одной нетерминальной вершины к другой, который не допускает язык.
  
  Если у нас есть переход переход от одной нетерминальной вершины к другой, который не допускает язык, то должно было быть соответствующее правило, но его нет.
  
  Поэтому алгоритм допускает только слова из языка}
  
 \digraph{51}{
        size="6,6";
        rankdir="LR";
        node [shape=point]; 0;
        node [shape=circle]; A C;
        node [shape=doublecircle]; B D;
        0 -> A;
        A -> B [label="a"];
        B -> B [label="a"];
        A -> C [label="b"];
        C -> A;
        C -> D [label="a"];
        D -> D [label="a|b"];
    }
    
\subsection{ Привести алгоритм построения КС грамматики по НКА. Доказать, что с помощью алгоритма мы можем получить только слова из языка НКА. Проиллюстрировать алгоритм на грамматике:}


{Алгоритм: }
\begin{itemize}
 \item {
    Нетерминалами  грамматики  будут  вершины  автомата, терминалами — пометки дуг
 }
 \item {
    Для  каждой дуги  из  вершины А в  вершину В,  помеченная  символом а в грамматику  добавляется правило $A \rightarrow aB$. Для каждой заключительной вершины В в грамматику добавляется правило $B \rightarrow \lambda$
 }
 
 \item {
Начальным  символом  будет  нетерминал,  соответствующий начальной  вершине.
 }
 
\end{itemize}

{Доказательство аналогично}

$$ q_0 \rightarrow aq_0 | aq_1 | q_3 $$
$$ q_1 \rightarrow aq_1 | aq_2 | q_2 | bq_4 $$
$$ q_2 \rightarrow aq_2 | bq_2 | aq_5 | \lambda$$
$$ q_3 \rightarrow bq_0 | \lambda$$
$$ q_4 \rightarrow q_5 | \lambda$$
$$ q_5 \rightarrow aq_5 | bq_2$$
$$ $$
$$ $$
\end{document}