\documentclass[a4paper,12pt]{article}
 
\usepackage{cmap}		
\usepackage[utf8]{inputenc}			
\usepackage[english,russian]{babel}
\usepackage{framed}
\usepackage{hyperref}
\usepackage{amsmath}
\usepackage{graphicx}
\usepackage[colorinlistoftodos]{todonotes}
\usepackage{wrapfig}
\usepackage{lipsum}
\usepackage{color}
\usepackage{indentfirst}
\usepackage{times}
\usepackage{textcomp}
\usepackage{float}
\usepackage{listings}
\usepackage{xcolor}
\usepackage[T1]{fontenc}
 
\usepackage{morewrites}
 
\usepackage[pdf]{graphviz}
\usepackage{xpatch}
\makeatletter
\newcommand*{\addFileDependency}[1]{% argument=file name and extension
  \typeout{(#1)}
  \@addtofilelist{#1}
  \IfFileExists{#1}{}{\typeout{No file #1.}}
}
\makeatother
\xpretocmd{\digraph}{\addFileDependency{#2.dot}}{}{}
 
\usepackage[autosize]{dot2texi}
\usepackage{tikz}
\usetikzlibrary{shapes,arrows}

\title{Домашнее задание №1}
\author{Сабитов Сергей А-13а-19}

\begin{document}
\maketitle


\section{Задание №1}
\subsection{$\{\omega \in (a,b,c)^* | |\omega|_{c} = 1\}$}
\digraph{011}{
        size="6,6";
        rankdir="LR";
        0 [shape=point];
        1 [shape=circle];
        2 [shape=doublecircle];
        0 -> 1;
        1 -> 1 [label="a, b"];
        1 -> 2 [label="c"];
        2 -> 2 [label="a, b"];
    }
\subsection{$\{\omega \in a,b^* |  |\omega|_{a} \leq 2,|\omega|_{b} \geq 2\}$}
\digraph{012}{
        size="6,6";
        rankdir="LR";
        0 [shape=point];
        1 [shape=circle];
        2 [shape=doublecircle];
        0 -> 1;
        1 -> 1 [label="a, b"];
        1 -> 2 [label="c"];
        2 -> 2 [label="a, b"];
    }
\subsection{$\{\omega \in (a,b)^* | |\omega|_{a} \neq |\omega|_{b}\}$}
{Конечный автомат не имеет паямяти, поэтому мы не можем поставить условие равенства или не равенства.}
\subsection{$\{\omega \in a,b^* |  \omega\omega = \omega\omega\omega\}$}
\begin{center}
\digraph{014}{
        size="6,6";
        rankdir="LR";
        0 [shape=point];
        1 [shape=doublecircle];
        0 -> 1;
    }

    Язык состоит из пустого слова, так как только для $\omega = \lambda$ условие выполняется
\end{center}

\section{Задание №2. Построить конечный автомат, используя прямое произведение}

\subsection{$L_{1} = \{\omega \in (a,b)| |\omega|_{a} \geq 2 \land |\omega|_{b} \geq 2 \}$}
\begin{center}
    $L_11 = \{\Sigma = \{a, b\}, Q_1 = \{1, 2, 3\}, 1, T_1 = \{3\}, \delta_1 \}$
 
    \digraph{0211}{
        size="6,6";
        rankdir="LR";
        node [shape=point]; 0;
        node [shape=circle]; 1 2;
        node [shape=doublecircle]; 3;
        0 -> 1;
        1 -> 1 [label="b"];
        1 -> 2 [label="a"];
        2 -> 2 [label="b"];
        2 -> 3 [label="a"];
        3 -> 3 [label="a, b"];
    }
 
    $L_12 = \{\Sigma = \{a, b\}, Q_2 = \{1, 2, 3\}, 1, T_2 = \{3\}, \delta_2 \}$
 
    \digraph{0212}{
        size="6,6";
        rankdir="LR";
        node [shape=point]; 0;
        node [shape=circle]; 1 2;
        node [shape=doublecircle]; 3;
        0 -> 1;
        1 -> 1 [label="a"];
        1 -> 2 [label="b"];
        2 -> 2 [label="a"];
        2 -> 3 [label="b"];
        3 -> 3 [label="a, b"];
    }
\end{center}
 
    \begin{enumerate}
        \item $\Sigma = \{a, b\}$
        \item $Q = \{(1, 1), (1, 2), (1, 3), (2, 1), (2, 2), (2, 3), (3, 1), (3, 2), (3, 3)\}$
        \item $S = \{(1, 1)\}$
        \item $T = \{(3, 3)\}$
        \item $\delta:$
    \end{enumerate}
    \begin{tabular}{|c|c|c|c|}
        \hline
        A1 & A2 & a & b  \\
        
        1 & 1 & 21 & 12  \\
        
        1 & 2 & 22 & 13  \\
        
        1 & 3 & 23 & 13  \\
        
        2 & 1 & 31 & 22  \\
        
        2 & 2 & 32 & 23  \\
        
        2 & 3 & 33 & 23  \\
        
        3 & 1 & 31 & 32  \\
        
        3 & 2 & 32 & 33  \\
        
        3 & 3 & 33 & 33  \\
        \hline
    \end{tabular}
 
    \begin{center}
        \digraph{021}{
            size="6,6";
            rankdir="LR";
            node [shape=point]; 0;
            node [shape=circle]; 11 12 13 21 22 23 31 32;
            node [shape=doublecircle]; 33;
            0 -> 11;
            11 -> 21 [label="a"];
            11 -> 12 [label="b"];
            12 -> 22 [label="a"];
            12 -> 13 [label="b"];
            13 -> 23 [label="a"];
            13 -> 13 [label="b"];
            21 -> 31 [label="a"];
            21 -> 22 [label="b"];
            22 -> 32 [label="a"];
            22 -> 23 [label="b"];
            23 -> 33 [label="a"];
            23 -> 23 [label="b"];
            31 -> 31 [label="a"];
            31 -> 32 [label="b"];
            32 -> 33 [label="a, b"];
            33 -> 33 [label="a, b"];
        }
    \end{center}
\subsection{$L_{2} = \{\omega \in (a,b)^*| |\omega| \geq 3 \land |\omega| \text{нечетное} \}$}
 Имеем следующие автоматы: \\ \\
    $|\omega| \geq 3$ \\ \\
    $L21 = \{\Sigma = \{a, b\}, Q_1 = \{1, 2, 3, 4\},S_1 =\{ 1\}, T_1 = \{4\}, \delta_1 \}$
    \newline
    \digraph[]{0221}{
        size="6,6";
        rankdir="LR";
        node [shape=point]; 0;
        node [shape=circle]; 1 2,3;
        node [shape=doublecircle]; 4;
        0 -> 1;
        1 -> 2 [label="a,b"];
        2 -> 3 [label="a,b"];
        3 -> 4 [label="a,b"];
        4 -> 4 [label="a,b"];
    }
    $|\omega|\:\texttt{нечетное}$ \\ \\ 
    $L22 = \{\Sigma = \{a, b\}, Q_1 = \{1, 2\},S_1 =\{ 1\}, T_1 = \{2\}, \delta_2 \}$
    \newline
    \digraph[]{0222}{
            size="6,6";
            rankdir="LR";
            0 [shape=point];
            1 [shape=circle];
            2 [shape=doublecircle];
            0 -> 1;
            1 -> 2 [label="a,b"];
            2 -> 1 [label="a,b"];
        }
Строим прямое произведение
        \begin{enumerate}
        \item $\Sigma = \{a, b\}$
        \item $Q = \{(1, 1), (1, 2), (2, 1), (2, 2), (3, 1), (3, 2), (4, 1), (4, 2), \}$
        \item $S = \{(1, 1)\}$
        \item $T = \{(4,2)\}$
        \item $\delta:$
 \begin{center}
      \begin{tabular}{ | c | c | c | c | }
\hline
$L21$ & $L22$ & a & b \\ \hline
 1 & 1 & 2,2 & 2,2 \\
 1 & 2 & 2,1 & 2,1 \\
 2 & 1 & 3,2 & 3,2 \\
 2 & 2 & 3,1 & 3,1 \\
 3 & 1 & 4,2 & 4,2 \\
 3 & 2 & 4,1 & 4,1 \\
 4 & 1 & 4,2 & 4,2 \\
 4 & 2 & 4,1 & 4,1 \\
\hline
\end{tabular} 
 \begin{center}
        \digraph[scale=0.8]{022}{
            size="6,6";
            rankdir="LR";
            node [shape=point]; 0;
            node [shape=circle]; 11, 12, 21, 22, 31, 32, 41;
            node [shape=doublecircle]; 42;
            0 -> 11;
            11 -> 22 [label="a,b"];
            12 -> 21 [label="a,b"];
            21 -> 32 [label="a,b"];
            22 -> 31 [label="a,b"];
            31 -> 42 [label="a,b"];
            32 -> 41 [label="a,b"];
            41 -> 42 [label="a,b"];
            42 -> 41 [label="a,b"];
        }
    \end{center}
\end{center}
    
\end{enumerate}
\subsection{$L_{3} = \{\omega \in (a,b)| |\omega|_{a} \text{четно} \land |\omega| \text{кратно трем}\}$}
$L_{31}=\{\omega \in (a,b)| |\omega|_{a} \text{четно}\} = \{\Sigma = \{a, b\}, Q_1 = \{1, 2\}, 1, T_1 = \{1\}, \delta_1 \}$
\\
\digraph{0231}{
        size="6,6";
        rankdir="LR";
        0 [shape=point];
        1 [shape=doublecircle];
        2 [shape=circle];
        0 -> 1;
        1 -> 1 [label="b"];
        1 -> 2 [label="a"];
        2 -> 2 [label="b"];
        2 -> 1 [label="a"]
    }
\\
$L_{32}=\{\omega \in (a,b)| |\omega|_{b} \text{кратно 3}\} = \{\Sigma = \{a, b\}, Q_1 = \{1, 2, 3\}, 1, T_1 = \{1\}, \delta_1 \}$
\\
\digraph{0232}{
        size="6,6";
        rankdir="LR";
        0 [shape=point];
        1 [shape=doublecircle];
        node [shape=circle]; 2 3;
        0 -> 1;
        1 -> 1 [label="a"];
        1 -> 2 [label="b"];
        2 -> 2 [label="a"];
        2 -> 3 [label="b"];
        3 -> 3 [label="a"];
        3 -> 1 [label="b"]
    }
\\
\begin{center}
    \begin{enumerate}
        \item $\Sigma = \{a, b\}$
        \item $Q = \{(1, 1), (1, 2), (1, 3), (2, 1), (2, 2), (2, 3)\}$
        \item $S = \{(1, 1)\}$
        \item $T = \{(1, 1)\}$
        \item $\delta:$
 \end{enumerate}
\begin{tabular}{ | c | c | c | c |}
\hline
L31 & L32 & a & b \\ \hline
1 & 1 & 21 & 12 \\
1 & 2 & 22 & 13 \\
1 & 3 & 23 & 11 \\
2 & 1 & 11 & 22 \\
2 & 2 & 12 & 23 \\
2 & 3 & 13 & 21 \\
\hline
\end{tabular}
\\
\digraph{023}{
        size="6,6";
        rankdir="LR";
        0 [shape=point];
        11 [shape=doublecircle];
        node [shape=circle]; 12 13 21 22 23;
        0 -> 11;
        11 -> 21 [label="a"];
        11 -> 12 [label="b"];
        12 -> 22 [label="a"];
        12 -> 13 [label="b"];
        13 -> 23 [label="a"];
        13 -> 11 [label="b"];
        21 -> 11 [label="a"];
        21 -> 22 [label="b"];
        22 -> 12 [label="a"];
        22 -> 23 [label="b"];
        23 -> 13 [label="a"];
        23 -> 21 [label="b"];
    }
\end{center}
\subsection{$L_{4} = \overline{L_{3}}$}
$\overline{L_{3}} = \{\Sigma_3,Q_3, s_3, Q_3\backslash T_3, \delta_3 \}$
$Q_3\backslash T_3 = 12, 13, 21, 22, 23$
\\
\digraph{024}{
        size="6,6";
        rankdir="LR";
        0 [shape=point];
        11 [shape=circle];
        node [shape=doublecircle]; 12 13 21 22 23;
        0 -> 11;
        11 -> 21 [label="a"];
        11 -> 12 [label="b"];
        12 -> 22 [label="a"];
        12 -> 13 [label="b"];
        13 -> 23 [label="a"];
        13 -> 11 [label="b"];
        21 -> 11 [label="a"];
        21 -> 22 [label="b"];
        22 -> 12 [label="a"];
        22 -> 23 [label="b"];
        23 -> 13 [label="a"];
        23 -> 21 [label="b"];
    }
\subsection{$L_{4} = L_{2} - L_{3}$}


\section{Задание №3}
\subsection{$\{ab+aba\}^*a$}
Автомат с лямбда переходами:
\\
\digraph{0311}{
        size="6,6";
        rankdir="LR";
        0 [shape=point];
        node [shape=circle];0 1 2 3 4 5 6 7 8 9;
        node [shape=doublecircle]; 10;
        0 -> 1;
        1 -> 2 [label="lambda"];
        1 -> 5 [label="lambda"];
        1 -> 9 [label="lambda"];
        2 -> 3 [label="a"];
        3 -> 4 [label="b"];
        4 -> 9 [label="lambda"];
        5 -> 6 [label="a"];
        6 -> 7 [label="b"];
        7 -> 8 [label="a"];
        8 -> 9 [label="lambda"];
        9 -> 10 [label="a"];
        9 -> 1 [label="lambda"];
    }
\newpage
Удаляя лямбда переходы и применяя алгоритм Томпсона получаем ДКА:
\\
\digraph{0312}{
        size="6,6";
        rankdir="LR";
        node [shape=point]; 0;
        node [shape=circle];1, 47;
        node [shape=doublecircle];3610, 36810;
        0 -> 1;
        1 -> 3610 [label="a"];
        3610 -> 47 [label="b"];
        47 -> 36810 [label="a"];
        36810 -> 47 [label="b"];
        36810 -> 3610 [label="a"];
    }
\subsection{$a(a(ab)^*b)^*(ab)^*$}
НКА: 
\newline
\digraph{0321}{
    size="6,6";
    rankdir="LR";
    0 [shape=point];
    node [shape=doublecircle]; 2 5;
    node [shape=circle]; 1 3 4 6;
    0 -> 1;
    1 -> 2 [label="a"];
    2 -> 3 [label="a"];
    3 -> 4 [label="a"];
    4 -> 3 [label="b"];
    3 -> 2 [label="b"];
    5 -> 6 [label = "a"];
    6 -> 5 [label = "b"];
    2 -> 5 [label = "lambda"];
}
\newline
ДКА:
\newline
\digraph{032}{
    size="6,6";
    rankdir="LR";
    0 [shape=point];
    2 [shape=doublecircle];
    node [shape=circle]; 1 3 4;
    0 -> 1;
    1 -> 2 [label="a"];
    2 -> 3 [label="a"];
    3 -> 4 [label="a"];
    4 -> 3 [label="b"];
    3 -> 2 [label="b"];
}

\subsection{$(a+(a+b)(a+b)b)^*$}
        \begin{enumerate}
        \item Строим НКА:
        \begin{center}
            \digraph{0331}{
                size="6,6";
                rankdir="LR";
                node [shape=point]; 0;
                node [shape=circle]; 2 3;
                node [shape=doublecircle]; 1;
                0 -> 1;
                1 -> 1 [label="a"];
                1 -> 2 [label="a, b"];
                2 -> 3 [label="a, b"];
                3 -> 1 [label="b"];
            }
        \end{center}
 
        \item По НКА строим эквивалентный ДКА (алгоритм Томпсона):
        \begin{center}
            \digraph[scale=0.7]{0332}{
                size="6,6";
                rankdir="LR";
                node [shape=point]; 0;
                node [shape=circle]; 2 3 23;
                node [shape=doublecircle]; 1 12 13 123;
                0 -> 1;
                1 -> 12 [label="a"];
                1 -> 2 [label="b"];
                2 -> 3 [label="a, b"];
                3 -> 1 [label="b"]
                12 -> 123 [label="a"];
                12 -> 23 [label="b"];
                123 -> 123 [label="a, b"];
                23 -> 3 [label="a"];
                23 -> 13 [label="b"];
                13 -> 12 [label="a, b"]
            }  
        \end{center}
 
    \end{enumerate}

\subsection{$(b+c)((ab)^*c+(ba)^*)^*$}
\digraph{034}{
        size="6,6";
        rankdir="LR";
        node [shape=point]; 0;
        node [shape=circle]; 2 3 4;
        node [shape=doublecircle]; 1;
        0 -> 1;
        1 -> 1 [label="c"];
        1 -> 2 [label="a"];
        2 -> 3 [label="b"];
        3 -> 2 [label="a"];
        3 -> 1 [label="c"];
        1 -> 4 [label="b"]
        4 -> 1 [label="a"]
	}
\subsection{Построим НКА}
\digraph[scale=0.9]{035}{
    size="6,6";
    rankdir=LR;
    node [shape=point]; 0;
    node [shape=doublecircle]; 13;
    node [shape=circle];
    0 -> 1;
    1 -> 2 [label = "a, b"];
    2 -> 1 [label = "a, b"];
    2 -> 3 [label = "a"];
    2 -> 8 [label = "b"];
    3 -> 4 [label = "a"];
    3 -> 5 [label = "b"];
    4 -> 13 [label = "a, b"];
    5 -> 6 [label = "a"];
    6 -> 7 [label = "b"];
    7 -> 13 [label = "a, b"];
    8 -> 9 [label = "b"];
    8 -> 10 [label = "a"];
    9 -> 13 [label = "a, b"];
    10 -> 11 [label = "b"];
    11 -> 12 [label = "a"];
    12 -> 13 [label = "a, b"];
    13 -> 13 [label = "a, b"];
}  
\includegraphics[scale=0.5]{мем.jpg}
\section{Задание №4}
\subsection{\{(aab)^*b(aba)^m | n \geq 0, m \geq 0\}}
Язык является регулярным, построим автомат:
\newline
\digraph[scale=0.8]{041}{
    size="6,6";
    rankdir="LR";
    node [shape=point]; 0;
    node [shape=circle]; 1 2 3 4 6 7;
    node [shape=doublecircle]; 5 8;
    0 -> 1;
    1 -> 2 [label="a"];
    2 -> 3 [label="a"];
    3 -> 4 [label="b"];
    4 -> 2 [label="a"];
    4 -> 5 [label="b"];
    5 -> 6 [label="a"]
    6 -> 7 [label="b"]
    7 -> 8 [label="a"];
    8 -> 6 [label="a"]
}
\subsection{\{$uaav | u \in (a,b)^*, v \in (a,b)^*, |u|_b\geq |u|_a$\}}
Рассмотрим отрицание языка L: $\overline{L}= \{uaav | u \in (a,b)^*, v \in (a,b)^*, |u|_b < |u|_a$\}
\newline
Фиксируем произвольное $n\in N$. Возьмем слово $w = b^naaa^n$.

Длина слова не меньше n: $|w|=2*n+2\geq n$.

Рассмотрим следующее разбиение слова:

    $$x=b^{k} \quad  y=b^{n-k} \quad$$ $$z=aaa^n$$
    
    $$y\neq \lambda \quad |xy|\leq n$$
    
    $$k\neq 0\quad n-k \geq 0$$

Других разбиений нет.
Если язык регулярный, то $\forall i \geq 0: xy^iz \in \overline{L}$

    $$b^{k} b^{(n-k)*i} aaa^{n}$$
    
    Видно, что $\forall i>1: xy^iz \notin \overline{L}$ 
    
    Так как  {$\overline{L}$} нерегулярный, то и L - нерегулярный язык.
\subsection{$L=\{a^m\omega | \omega \in \{a,b\}^*, 1\leq |\omega|_b\leq m\}$}
Рассмотрим отрицание языка L: $\overline{L}=\{a^m\omega | \omega \in \{a,b\}^*, |\omega|_b\geq m\}$
\newline
Фиксируем произвольное $n\in N$. Возьмем слово $w = a^nb^n$.
\newline
Длина слова не меньше n: $|w|=2*n\geq n$.
\newline
Рассмотрим следующее разбиение слова:
\begin{center}
    $$x=a^{k_1} \quad  y=a^{k_2} \quad$$ $$z=a^{n-k_1-k_2}b^n$$
    
    $$y\neq \lambda \quad |xy|\leq n$$
    
    $$k_1\neq 0 \quad k_1+k_2\leq n$$
\end{center}
Других разбиений нет.
Если язык регулярный, то $\forall i \geq 0: xy^iz \in \overline{L}$

    $$a^{k_1} a^{k_2*i} a^{n-k_1-k_2}b^n = a^{n+k_1*(i-1)}b^n$$
    
    Видно, что $\exists i: xy^iz \notin \overline{L}$ 
    
    Так как  {$\overline{L}$} нерегулярный, то и L - нерегулярный язык.
\subsection{$L=\{a^kb^ma^n | k = n\vee m > 0\}$}
Фиксируем произвольное $n\in N$.\newline 
Берем слово\newline  $$w = a^{n-1}ba^n\in L$$\newline
Очевидно, длина слова не меньше n: $|w|=2n\geq n$.
\newline
Рассмотрим следующее разбиение слова:
$$ a^{n-1-l}a^lba^n$$
где $x=a^{n-1-l}$;\:$y=a^{l}b$;\: $z=a^{n}$ и $l \geq 0;\:l \leq n-1;$\newline
Других разбиений нет.\newline
При $i = 0$ слово $xy^iz = a^{n-1-l}(a^{l}b)^ia^{n} = a^{n-1-l}a^n  $ не будет пренадлежать языку $L$, следовательно, язык  нерегулярный.\newline
\subsection{$L=\{ucv|\: u \in \{a,b\}^*\:v \in \{a,b\}^* u \neq v^R\}$}
Фиксируем произвольное $n\in N$.

Берем слово

$$w =a^nca^{2n}\in L$$\newline
Длина слова не меньше n: $|w|=3n+1\geq n$.

Рассмотрим следующее разбиение слова:
$$ a^{n-l}a^lca^{2n}$$
где $x=a^{n-l}$;\:$y=a^{l}$;\: $z=ca^{2n}$ и $l > 0;\:l \leq n;$

Других разбиений нет.

При $i = 2$ слово принимает такой вид: $xy^iz = a^{n-l}(a^{l})^ica^{2n} = a^{n-l+il}ca^n = a^{n+l}ca^{2n}  $;\newline
Таким образом, при $l = n$:

$$a^{2n}ca^{2n}$$То есть, $u = v^R $, следовательно, слово не будет пренадлежать языку $L$, следовательно, язык  $L$ - нерегулярный.\newline
\end{document}